\RequirePackage[l2tabu,orthodox]{nag}
\documentclass[12pt,letterpaper]{article}%
    % Add ``,titlepage'' after ``letterpaper'' to get a separate page
    % for the title information.
%%%%%%%%%%%%%%%%%%%% Packages
\usepackage[T1]{fontenc}
\usepackage[utf8]{inputenc}
%\usepackage{xeCJK}
%\setCJKmainfont{IPAexMincho}
    % Be sure to use XeLaTeX to compile. Since the ``microtype'' package
    % only works with pdflatex, make sure to disable that too.
\usepackage[margin=1in]{geometry}%
    % Use 1 inch margins (all around).
%\usepackage{setspace}%
    % For double spacing, etc. See ``\doublespacing'' and
    % ``\onehalfspacing'' below.
%\usepackage[parfill]{parskip}%
    % New paragraphs get an empty line rather than an indent.
%\usepackage{soul}%
    % Use ``\ul{text}'' to underline text.
\usepackage{amsmath}
\usepackage{mathtools}
\usepackage{amsfonts}
\usepackage{amssymb}
\usepackage{mathrsfs}%
    % Example: ``\mathscr{A}''.
%\usepackage{amsthm}%
    % Enable this and ``thmtools'' to get theorem-like environments.
%\usepackage{thmtools}
    % Note: ``qed=...'' only works with TeXLive2011 or newer (I think).
    % For older distributions, just remove ``qed=...''. Also, theorem
    % declarations must come before ``hyperref''.
%\declaretheorem[style=definition,qed=\(\lrcorner\),parent=subsection]%
    %{axiom}
%\declaretheorem[style=definition,qed=\(\lrcorner\),sibling=axiom]%
    %{corollary}
%\declaretheorem[style=definition,qed=\(\lrcorner\),sibling=axiom]%
    %{definition}
%\declaretheorem[style=definition,qed=\(\lrcorner\),sibling=axiom]%
    %{discussion}
%\declaretheorem[style=definition,qed=\(\lrcorner\),sibling=axiom]%
    %{example}
%\declaretheorem[style=definition,qed=\(\lrcorner\),sibling=axiom]%
    %{exercise}
%\declaretheorem[style=definition,qed=\(\lrcorner\),sibling=axiom]%
    %{lemma}
%\declaretheorem[style=definition,qed=\(\lrcorner\),sibling=axiom]%
    %{note}
%\declaretheorem[style=definition,qed=\(\lrcorner\),sibling=axiom]%
    %{proposition}
%\declaretheorem[style=definition,qed=\(\lrcorner\),sibling=axiom]%
    %{remark}
%\declaretheorem[style=definition,qed=\(\lrcorner\),sibling=axiom]%
    %{theorem}
%\usepackage{siunitx}%
    % For units.
\usepackage[all,error]{onlyamsmath}
\usepackage{fixltx2e}
\usepackage{booktabs}
%\usepackage{multirow}
\usepackage{array}
\usepackage{paralist}
\usepackage{verbatim}
\usepackage{subfig}
\usepackage{graphicx}
%\usepackage{textcomp}%
    % To obtain straight single (') and double (") quotes (instead of
    % curly ones); use ``\textquotesingle'' and ``\textquotedbl''.
    % Note however that ``\textquotedbl'' comes from the fontenc
    % package.
%%%%%%%%%%%%%%%%%%%% Displaying source code
%\usepackage{listings}%
    % For displaying source code. Use
    % ``\lstinline[language=tex]!code here!'' to display (TeX) code
    % within a paragraph. Use
    % ``\begin{lstlisting}[language=tex]code here\end{lstlisting}'' to
    % show (TeX) code as its own paragraph. Use
    % ``\lstinputlisting[language=tex]{file.tex}'' to display (TeX)
    % code from a separate file (``file.tex'').
%\lstset%
%{%
    %basicstyle=\ttfamily,%
    %keywordstyle=\bfseries,%
    %commentstyle=\slshape,%
    %stringstyle=\ttfamily,%
    %showspaces=false,%
    %showstringspaces=false,%
    %upquote=true,%
        % Don't make quotes curly.
    %numbers=left%
%}%
\usepackage[tone]{tipa}%
    % For the International Phonetic Alphabet. Example:
    % ``\textipa{text}''.
%\usepackage{tikz}
%\usetikzlibrary{decorations.markings}
%\usetikzlibrary{snakes}
%%%%%%%%%%%%%%%%%%%% Fonts
    % Some fonts with (at least partial) maths support.
\usepackage{lmodern}%
    % Computer Modern, but apparently enhanced.
%\usepackage{mathptmx}%
    % Times New Roman.
%\usepackage{fouriernc}
%\usepackage{mathpazo}
%\usepackage{bookman}
%\usepackage{pxfonts}
%\usepackage[utopia]{mathdesign}
%\usepackage[charter]{mathdesign}
%%%%%%%%%%%%%%%%%%%% Packages (continued)
\usepackage{microtype}
%\usepackage{hyperref}%
    % Must be last package, except for ``ellipsis'' and ``cleveref''.
\usepackage{cleveref}%
    % Must be after ``hyperref''.
\usepackage{ellipsis}
    % Must be after ``hyperref'' (and probably ``cleveref'').
%%%%%%%%%%%%%%%%%%%% Document information
%\title{TitleOfDocument}
%\author{AuthorOfDocument}
%\date{\today{}}%
    % Uncomment and leave this empty to have no date.
%%%%%%%%%%%%%%%%%%%% Custom commands
\newcommand{\definiendum}[1]{\textbf{#1}}%
    % Example: ``A \definiendum{group} is a ...''.
    % Also see the Wikipedia articles ``Separation of concerns'' and
    % ``Separation of presentation and content''.
%%%%%%%%%%%%%%%%%%%% For maths
%\crefname{axiom}{axiom}{axioms}
%\crefname{corollary}{corollary}{corollaries}
%\crefname{definition}{definition}{definitions}
%\crefname{discussion}{discussion}{discussions}
%\crefname{example}{example}{examples}
%\crefname{exercise}{exercise}{exercises}
%\crefname{lemma}{lemma}{lemmas}
%\crefname{note}{note}{notes}
%\crefname{proposition}{proposition}{propositions}
%\crefname{remark}{remark}{remarks}
%\crefname{theorem}{theorem}{theorems}
    % Abbreviations for sets of numbers.
%\newcommand{\N}{\ensuremath{\mathbf{N}}}
%\newcommand{\Z}{\ensuremath{\mathbf{Z}}}
%\newcommand{\Q}{\ensuremath{\mathbf{Q}}}
%\newcommand{\R}{\ensuremath{\mathbf{R}}}
%\newcommand{\C}{\ensuremath{\mathbf{C}}}
%%%%%%%%%%%%%%%%%%%% Other options
%\doublespacing%
    % Use with the ``setspace'' package.
%\onehalfspacing
    % Use with the ``setspace'' package.
%%%%%%%%%%%%%%%%%%%% Begin document
\begin{document}
%\maketitle
%\tableofcontents
%\newpage

\newcommand{\phon}[1]{/\textipa{#1}/}
\newcommand{\barsmallI}{\ipabar{\textsci}{.5ex}{1.1}{}{}}
    % See http://www.tug.org/tugboat/tb17-2/tb51rei.pdf
    % p 113
%\newcommand{\barsmallI}{1}

\noindent\emph{The Greek Alphabet for Mathematics and the Sciences}

From new research.

References:

\begin{itemize}
    \item OED: Oxford English Dictionary. Online.
\end{itemize}

\begin{description}
    \item[Alpha] US: \phon{"\ae lf@} (OED: 3rd ed 2012)

                Brit.: \phon{"alf@} (OED: 3rd ed 2012)

    \item[Beta] \phon{"bi:t@} (OED: 2nd ed 1989 unrevised)

    \item[Gamma] \phon{"g\ae m@} (OED: 2nd ed 1989 unrevised)

    \item[Delta] \phon{"dElt@} (OED: 2nd ed 1989 unrevised)

    \item[Epsilon] \phon{Ep"saIl@n} (OED: 2nd ed 1989 unrevised)

    \item[Zeta] \phon{"zi:t@} (OED: 2nd ed 1989 unrevised)

    \item[Eta] \phon{"i:t@} (OED: 2nd ed 1989 unrevised)

                \phon{"eIt@} (OED: 2nd ed 1989 unrevised)

    \item[Theta] \phon{"Ti:t@} (OED: 2nd ed 1989 unrevised)

    \item[Iota] \phon{aI"@Ut@} (OED: 2nd ed 1989 unrevised)

    \item[Kappa] \phon{"k\ae p@} (OED: 2nd ed 1989 unrevised)

    \item[Lambda] \phon{"l\ae md@} (OED: 2nd ed 1989 unrevised)

    \item[Mu] Brit.: \phon{mju:} (OED: 3rd ed 2003)

                US: \phon{mju} (OED: 3rd ed 2003)

    \item[Nu] Brit.: \phon{nju:} (OED: 3rd ed 2003)

                US: \phon{nu} (OED: 3rd ed 2003)

                US: \phon{nju} (OED: 3rd ed 2003)

    \item[Xi] \phon{saI} (OED: 2nd ed 1989 unrevised)

                \phon{zaI} (OED: 2nd ed 1989 unrevised)

                \phon{ksaI} (OED: 2nd ed 1989 unrevised)

                \phon{gzaI} (OED: 2nd ed 1989 unrevised)

    \item[Omicron] Brit.: \phon{@"m2Ikr6n} (OED: 3rd ed 2004)

                    Brit.: \phon{@U"m2Ikr6n} (OED: 3rd ed 2004)

                    Brit.: \phon{@"m2Ikrn} (OED: 3rd ed 2004)

                    Brit.: \phon{@"m2Ikr@n} (OED: 3rd ed 2004)

                    Brit.: \phon{@"Um2Ikrn} (OED: 3rd ed 2004)

                    Brit.: \phon{@"Um2Ikr@n} (OED: 3rd ed 2004)

                    Brit.: \phon{"6m\barsmallI kr6n} (OED: 3rd ed 2004)

                    Brit.: \phon{"6m\barsmallI krn} (OED: 3rd ed 2004)

                    Brit.: \phon{"6m\barsmallI kr@n} (OED: 3rd ed 2004)

                    US: \phon{"Am@""krAn} (OED: 3rd ed 2004)

                    US: \phon{"oUm@""krAn} (OED: 3rd ed 2004)

    \item[Pi] Brit.: \phon{p2I} (OED: 3rd ed 2006)

                    US: \phon{paI} (OED: 3rd ed 2006)

    \item[Rho] Brit.: \phon{r@U} (OED: 3rd ed 2010)

                US: \phon{roU} (OED: 3rd ed 2010)

    \item[Sigma] \phon{"sIgm@} (OED: 2nd ed 1989 unrevised)

    \item[Tau] \phon{tO:} (OED: 2nd ed 1989 unrevised)

                \phon{taU} (OED: 2nd ed 1989 unrevised)

    \item[Upsilon] \phon{ju:p"saIl@n} (OED: 2nd ed 1989 unrevised)

                    \phon{"UpsIl6n} (OED: 2nd ed 1989 unrevised)

    \item[Phi] Brit.: \phon{f2I} (OED: 3rd ed 2005)

                US: \phon{faI} (OED: 3rd ed 2005)

    \item[Chi] \phon{kaI} (OED: 2nd ed 1989 unrevised)

    \item[Psi] Brit.: \phon{s2I} (OED: 3rd ed 2007)

               Brit.: \phon{ps2I} (OED: 3rd ed 2007)

               US: \phon{saI} (OED: 3rd ed 2007)

               US: \phon{psaI} (OED: 3rd ed 2007)

    \item[Omega] Brit.: \phon{"@Um\barsmallI g@} (OED: 3rd ed 2004)

                 US: \phon{oU"meIg@} (OED: 3rd ed 2004)

                 US: \phon{oU"mEg@} (OED: 3rd ed 2004)

\end{description}


\end{document}
%%%%%%%%%%%%%%%%%%%% End document
