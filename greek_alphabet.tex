\documentclass{article}
\usepackage[margin=2cm]{geometry}
\usepackage{amsmath}
\usepackage{tipa}
\usepackage{ragged2e}
\usepackage[colorlinks,urlcolor=blue]{hyperref}

\begin{document}
\RaggedRight
\emph{The Greek Alphabet for Mathematics and the Sciences.}

\begin{tabular}{l p{2.6cm} l p{9cm}}
%%%%%%%%%% Top Row %%%%%%%%%%
\textbf{Capital (\LaTeX)}\footnotemark[1] &
\textbf{Small (\LaTeX)}\footnotemark[1] &
\textbf{English} &
\textbf{Pronunciation\footnotemark[2]\ [IPA] (Approximation)}\\
%%%%%%%%%% Alpha %%%%%%%%%%
\(A\) (\verb|A|) &
\(\alpha\) (\verb|\alpha|) &
Alpha &
[\textipa{"\ae lf@}] (\textsc{al}-fuh)\\
%%%%%%%%%% Beta %%%%%%%%%%
\(B\) (\verb|B|) &
\(\beta\) (\verb|\beta|) &
Beta &
US: [\textipa{"beIt@}] (\textsc{bey}-tuh);
UK: [\textipa{"bi:t@}] (\textsc{bee}-tuh)\\
%%%%%%%%%% Gamma %%%%%%%%%%
\(\Gamma\) (\verb|\Gamma|) &
\(\gamma\) (\verb|\gamma|) &
Gamma &
[\textipa{"g\ae m@}] (\textsc{gam}-uh)\\
%%%%%%%%%% Delta %%%%%%%%%%
\(\Delta\) (\verb|\Delta|) &
\(\delta\) (\verb|\delta|) &
Delta &
[\textipa{"dElt@}] (\textsc{del}-tuh)\\
%%%%%%%%%% Epsilon %%%%%%%%%%
\(E\) (\verb|E|) &
\(\epsilon\) (\verb|\epsilon|), \(\varepsilon\) (\verb|\varepsilon|) &
Epsilon &
[\textipa{"EpsI""l6n}] (\textsc{ep}-sih-lon),
[\textipa{"Eps@""l6n}] (\textsc{ep}-suh-lon),
[\textipa{"Eps@""l@n}] (\textsc{ep}-suh-luhn);
UK: [\textipa{Ep"saIl@n}] (ep-\textsc{sahy}-luhn)\\
%%%%%%%%%% Zeta %%%%%%%%%%
\(Z\) (\verb|Z|) &
\(\zeta\) (\verb|\zeta|) &
Zeta &
US: [\textipa{"zeIt@}] (\textsc{zey}-tuh);
UK: [\textipa{"zi:t@}] (\textsc{zee}-tuh)\\
%%%%%%%%%% Eta %%%%%%%%%%
\(H\) (\verb|H|) &
\(\eta\) (\verb|\eta|) &
Eta &
US: [\textipa{"eIt@}] (\textsc{ey}-tuh);
UK: [\textipa{"i:t@}] (\textsc{ee}-tuh)\\
%%%%%%%%%% Theta %%%%%%%%%%
\(\Theta\) (\verb|\Theta|) &
\(\theta\) (\verb|\theta|), \(\vartheta\) (\verb|\vartheta|) &
Theta &
US: [\textipa{"TeIt@}] (\textsc{they}-tuh)\footnotemark[3];
UK: [\textipa{"Ti:t@}] (\textsc{thee}-tuh)\footnotemark[3]\\
%%%%%%%%%% Iota %%%%%%%%%%
\(I\) (\verb|I|) &
\(\iota\) (\verb|\iota|) &
Iota &
US: [\textipa{aI"oUt@}] (ahy-\textsc{ou}-tuh);
UK: [\textipa{aI"@Ut@}] (ahy-\textsc{uh.u}-tuh)\footnotemark[4]\\
%%%%%%%%%% Kappa %%%%%%%%%%
\(K\) (\verb|K|) &
\(\kappa\) (\verb|\kappa|) &
Kappa &
[\textipa{"k\ae p@}] (\textsc{kap}-uh)\\
%%%%%%%%%% Lambda %%%%%%%%%%
\(\Lambda\) (\verb|\Lambda|) &
\(\lambda\) (\verb|\lambda|) &
Lambda &
[\textipa{"l\ae md@}] (\textsc{lam}-duh)\\
%%%%%%%%%% Mu %%%%%%%%%%
\(M\) (\verb|M|) &
\(\mu\) (\verb|\mu|) &
Mu &
[\textipa{mju:}] (myoo),
[\textipa{mu:}] (moo)\\
%%%%%%%%%% Nu %%%%%%%%%%
\(N\) (\verb|N|) &
\(\nu\) (\verb|\nu|) &
Nu &
US: [\textipa{nu:}] (noo);
UK: [\textipa{nju:}] (nyoo)\\
%%%%%%%%%% Xi %%%%%%%%%%
\(\Xi\) (\verb|\Xi|) &
\(\xi\) (\verb|\xi|) &
Xi &
[\textipa{zaI}] (zahy),
[\textipa{saI}] (sahy),
[\textipa{ksaI}] (ksahy)\\
%%%%%%%%%% Omicron %%%%%%%%%%
\(O\) (\verb|O|) &
\(o\) (\verb|o|) &
Omicron &
[\textipa{"6mI""k\*r6n}] (\textsc{om}-ih-kron),
[\textipa{"oUmI""k\*r6n}] (\textsc{oh}-mih-kron);
UK: [\textipa{@U"maIk\*r6n}] (uh.u-\textsc{mahy}-kron)\footnotemark[4]\\
%%%%%%%%%% Pi %%%%%%%%%%
\(\Pi\) (\verb|\Pi|) &
\(\pi\) (\verb|\pi|), \(\varpi\) (\verb|\varpi|) &
Pi &
[\textipa{paI}] (pahy)\\
%%%%%%%%%% Rho %%%%%%%%%%
\(P\) (\verb|P|) &
\(\rho\) (\verb|\rho|), \(\varrho\) (\verb|\varrho|) &
Rho &
US: [\textipa{\*roU}] (roh);
UK: [\textipa{\*r@U}] (ruh.u)\footnotemark[4]\\
%%%%%%%%%% Sigma %%%%%%%%%%
\(\Sigma\) (\verb|\Sigma|) &
\(\sigma\) (\verb|\sigma|), \(\varsigma\) (\verb|\varsigma|) &
Sigma &
[\textipa{"sIgm@}] (\textsc{sig}-muh)\\
%%%%%%%%%% Tau %%%%%%%%%%
\(T\) (\verb|T|) &
\(\tau\) (\verb|\tau|) &
Tau &
[\textipa{taU}] (tau),
[\textipa{tO:}] (taw)\\
%%%%%%%%%% Upsilon %%%%%%%%%%
\(\Upsilon\) (\verb|\Upsilon|) &
\(\upsilon\) (\verb|\upsilon|) &
Upsilon &
[\textipa{"2psI""l6n}] (\textsc{up}-sih-lon),
[\textipa{"2ps@""l6n}] (\textsc{up}-suh-lon),
[\textipa{"@ps@""l6n}] (\textsc{uhp}-suh-lon),
[\textipa{"UpsI""l6n}] (\textsc{uup}-sih-lon),
[\textipa{"ju:ps@""l6n}] (\textsc{yoop}-suh-lon),
[\textipa{"ju:psI""l6n}] (\textsc{yoop}-sih-lon),
[\textipa{"ju:ps@""l@n}] (\textsc{yoop}-suh-luhn),
[\textipa{"u:ps@""l6n}] (\textsc{oop}-suh-lon);
UK: [\textipa{ju:p"saIl@n}] (yoop-\textsc{sahy}-luhn),
[\textipa{2p"saIl@n}] (up-\textsc{sahy}-luhn)\\
%%%%%%%%%% Phi %%%%%%%%%%
\(\Phi\) (\verb|\Phi|) &
\(\phi\) (\verb|\phi|), \(\varphi\) (\verb|\varphi|) &
Phi &
[\textipa{faI}] (fahy),
[\textipa{fi:}] (fee) \\
%%%%%%%%%% Chi %%%%%%%%%%
\(X\) (\verb|X|) &
\(\chi\) (\verb|\chi|) &
Chi &
[\textipa{kaI}] (kahy)\\
%%%%%%%%%% Psi %%%%%%%%%%
\(\Psi\) (\verb|\Psi|) &
\(\psi\) (\verb|\psi|) &
Psi &
[\textipa{saI}] (sahy),
[\textipa{psaI}] (psahy)\\
%%%%%%%%%% Omega %%%%%%%%%%
\(\Omega\) (\verb|\Omega|) &
\(\omega\) (\verb|\omega|) &
Omega &
US: [\textipa{oU"meIg@}] (oh-\textsc{mey}-guh),
[\textipa{oU"mi:g@}] (oh-\textsc{mee}-guh),
[\textipa{oU"mEg@}] (oh-\textsc{meg}-uh);
UK: [\textipa{"@UmI""g@}] (\textsc{uh.u}-mig-uh)\footnotemark[4],
[\textipa{"@UmE""g@}] (\textsc{uh.u}-meg-uh)\footnotemark[4]
\end{tabular}

\footnotetext[1]{These symbols only work in the mathematics environment.}

\footnotetext[2]{The pronunciations used in this document were checked for accuracy against

\begin{itemize}
    \item The entry of each letter on
    \begin{itemize}
        \item\href{http://dictionary.reference.com/}{Dictionary.com (Online)},
        \item\href{http://en.wiktionary.org/wiki/Wiktionary:Main_Page}{Wiktionary (Online)},
        \item \href{http://en.wikipedia.org/wiki/Main_Page}{Wikipedia (Online)},
        \item\href{http://www.collinsdictionary.com/dictionary/english}{Collins English Dictionary (Online)},
        \item \href{http://oxforddictionaries.com/}{Oxford Dictionaries (Online)},
        \item New Oxford American Dictionary (accessed through Apple's Dictionary software),
        \item The Merriam-Webster Dictionary (Print),
    \end{itemize}
    \item \href{http://en.wikipedia.org/wiki/English_pronunciation_of_Greek_letters}{The Wikipedia entry for ``English pronunciation of Greek letters''}, and
    \item \url{http://www.valdosta.edu/~cbarnbau/personal/teaching_MISC/greek.htm}.
\end{itemize}}

\footnotetext[3]{The ``th'' is soft as in the English ``thin'', not as in ``they''.}

\footnotetext[4]{As evident in the IPA transcription, ``uh.u'' represents a schwa immediately followed by a near-close near-back rounded vowel (found in the English ``put''), and is equivalent to the UK ``long O''. The dot was infixed so as not to be misinterpreted as [\textipa{UhU}].}
\end{document}
